%% Generated by Sphinx.
\def\sphinxdocclass{jupyterBook}
\documentclass[letterpaper,10pt,english]{jupyterBook}
\ifdefined\pdfpxdimen
   \let\sphinxpxdimen\pdfpxdimen\else\newdimen\sphinxpxdimen
\fi \sphinxpxdimen=.75bp\relax
\ifdefined\pdfimageresolution
    \pdfimageresolution= \numexpr \dimexpr1in\relax/\sphinxpxdimen\relax
\fi
%% let collapsible pdf bookmarks panel have high depth per default
\PassOptionsToPackage{bookmarksdepth=5}{hyperref}
%% turn off hyperref patch of \index as sphinx.xdy xindy module takes care of
%% suitable \hyperpage mark-up, working around hyperref-xindy incompatibility
\PassOptionsToPackage{hyperindex=false}{hyperref}
%% memoir class requires extra handling
\makeatletter\@ifclassloaded{memoir}
{\ifdefined\memhyperindexfalse\memhyperindexfalse\fi}{}\makeatother

\PassOptionsToPackage{warn}{textcomp}

\catcode`^^^^00a0\active\protected\def^^^^00a0{\leavevmode\nobreak\ }
\usepackage{cmap}
\usepackage{fontspec}
\defaultfontfeatures[\rmfamily,\sffamily,\ttfamily]{}
\usepackage{amsmath,amssymb,amstext}
\usepackage{polyglossia}
\setmainlanguage{english}



\setmainfont{FreeSerif}[
  Extension      = .otf,
  UprightFont    = *,
  ItalicFont     = *Italic,
  BoldFont       = *Bold,
  BoldItalicFont = *BoldItalic
]
\setsansfont{FreeSans}[
  Extension      = .otf,
  UprightFont    = *,
  ItalicFont     = *Oblique,
  BoldFont       = *Bold,
  BoldItalicFont = *BoldOblique,
]
\setmonofont{FreeMono}[
  Extension      = .otf,
  UprightFont    = *,
  ItalicFont     = *Oblique,
  BoldFont       = *Bold,
  BoldItalicFont = *BoldOblique,
]



\usepackage[Bjarne]{fncychap}
\usepackage[,numfigreset=1,mathnumfig]{sphinx}

\fvset{fontsize=\small}
\usepackage{geometry}


% Include hyperref last.
\usepackage{hyperref}
% Fix anchor placement for figures with captions.
\usepackage{hypcap}% it must be loaded after hyperref.
% Set up styles of URL: it should be placed after hyperref.
\urlstyle{same}


\usepackage{sphinxmessages}



        % Start of preamble defined in sphinx-jupyterbook-latex %
         \usepackage[Latin,Greek]{ucharclasses}
        \usepackage{unicode-math}
        % fixing title of the toc
        \addto\captionsenglish{\renewcommand{\contentsname}{Contents}}
        \hypersetup{
            pdfencoding=auto,
            psdextra
        }
        % End of preamble defined in sphinx-jupyterbook-latex %
        

\title{Metrology Lecture Book}
\date{May 09, 2022}
\release{}
\author{Katharina\sphinxhyphen{}S.\@{} Isleif}
\newcommand{\sphinxlogo}{\vbox{}}
\renewcommand{\releasename}{}
\makeindex
\begin{document}

\pagestyle{empty}
\sphinxmaketitle
\pagestyle{plain}
\sphinxtableofcontents
\pagestyle{normal}
\phantomsection\label{\detokenize{intro::doc}}


\sphinxAtStartPar
This is a small sample book to give you a feel for how book content is
structured.
It shows off a few of the major file types, as well as some sample content.
It does not go in\sphinxhyphen{}depth into any particular topic \sphinxhyphen{} check out \sphinxhref{https://jupyterbook.org}{the Jupyter Book documentation} for more information.

\sphinxAtStartPar
Check out the content pages bundled with this sample book to see more.
\begin{itemize}
\item {} 
\sphinxAtStartPar
{\hyperref[\detokenize{markdown::doc}]{\sphinxcrossref{Markdown Files}}}

\item {} 
\sphinxAtStartPar
{\hyperref[\detokenize{notebooks::doc}]{\sphinxcrossref{Content with notebooks}}}

\item {} 
\sphinxAtStartPar
{\hyperref[\detokenize{markdown-notebooks::doc}]{\sphinxcrossref{Notebooks with MyST Markdown}}}

\end{itemize}


\chapter{Markdown Files}
\label{\detokenize{markdown:markdown-files}}\label{\detokenize{markdown::doc}}
\sphinxAtStartPar
Whether you write your book’s content in Jupyter Notebooks (\sphinxcode{\sphinxupquote{.ipynb}}) or
in regular markdown files (\sphinxcode{\sphinxupquote{.md}}), you’ll write in the same flavor of markdown
called \sphinxstylestrong{MyST Markdown}.
This is a simple file to help you get started and show off some syntax.


\section{What is MyST?}
\label{\detokenize{markdown:what-is-myst}}
\sphinxAtStartPar
MyST stands for “Markedly Structured Text”. It
is a slight variation on a flavor of markdown called “CommonMark” markdown,
with small syntax extensions to allow you to write \sphinxstylestrong{roles} and \sphinxstylestrong{directives}
in the Sphinx ecosystem.

\sphinxAtStartPar
For more about MyST, see \sphinxhref{https://jupyterbook.org/content/myst.html}{the MyST Markdown Overview}.


\section{Sample Roles and Directivs}
\label{\detokenize{markdown:sample-roles-and-directivs}}
\sphinxAtStartPar
Roles and directives are two of the most powerful tools in Jupyter Book. They
are kind of like functions, but written in a markup language. They both
serve a similar purpose, but \sphinxstylestrong{roles are written in one line}, whereas
\sphinxstylestrong{directives span many lines}. They both accept different kinds of inputs,
and what they do with those inputs depends on the specific role or directive
that is being called.

\sphinxAtStartPar
Here is a “note” directive:

\begin{sphinxadmonition}{note}{Note:}
\sphinxAtStartPar
Here is a note
\end{sphinxadmonition}

\sphinxAtStartPar
It will be rendered in a special box when you build your book.

\sphinxAtStartPar
Here is an inline directive to refer to a document: {\hyperref[\detokenize{markdown-notebooks::doc}]{\sphinxcrossref{\DUrole{doc}{Notebooks with MyST Markdown}}}}.

\begin{figure}[htbp]
\centering
\capstart

\noindent\sphinxincludegraphics[width=200\sphinxpxdimen]{{logo}.png}
\caption{This is a caption in \sphinxstylestrong{Markdown}!}\label{\detokenize{markdown:markdown-fig}}\end{figure}


\section{Citations}
\label{\detokenize{markdown:citations}}
\sphinxAtStartPar
You can also cite references that are stored in a \sphinxcode{\sphinxupquote{bibtex}} file. For example,
the following syntax: \sphinxcode{\sphinxupquote{\{cite\}`holdgraf\_evidence\_2014`}} will render like
this: {[}\hyperlink{cite.markdown:id3}{HdHPK14}{]}.

\sphinxAtStartPar
Moreover, you can insert a bibliography into your page with this syntax:
The \sphinxcode{\sphinxupquote{\{bibliography\}}} directive must be used for all the \sphinxcode{\sphinxupquote{\{cite\}}} roles to
render properly.
For example, if the references for your book are stored in \sphinxcode{\sphinxupquote{references.bib}},
then the bibliography is inserted with:


\section{Learn more}
\label{\detokenize{markdown:learn-more}}
\sphinxAtStartPar
This is just a simple starter to get you started.
You can learn a lot more at \sphinxhref{https://jupyterbook.org}{jupyterbook.org}.


\chapter{Content with notebooks}
\label{\detokenize{notebooks:content-with-notebooks}}\label{\detokenize{notebooks::doc}}
\sphinxAtStartPar
You can also create content with Jupyter Notebooks. This means that you can include
code blocks and their outputs in your book.


\section{Markdown + notebooks}
\label{\detokenize{notebooks:markdown-notebooks}}
\sphinxAtStartPar
As it is markdown, you can embed images, HTML, etc into your posts!

\sphinxAtStartPar
\sphinxincludegraphics{{logo-wide}.png}

\sphinxAtStartPar
You can also \(add_{math}\) and
\begin{equation*}
\begin{split}
math^{blocks}
\end{split}
\end{equation*}
\sphinxAtStartPar
or
\begin{equation*}
\begin{split}
\begin{aligned}
\mbox{mean} la_{tex} \\ \\
math blocks
\end{aligned}
\end{split}
\end{equation*}
\sphinxAtStartPar
But make sure you \$Escape \$your \$dollar signs \$you want to keep!


\section{MyST markdown}
\label{\detokenize{notebooks:myst-markdown}}
\sphinxAtStartPar
MyST markdown works in Jupyter Notebooks as well. For more information about MyST markdown, check
out \sphinxhref{https://jupyterbook.org/content/myst.html}{the MyST guide in Jupyter Book},
or see \sphinxhref{https://myst-parser.readthedocs.io/en/latest/}{the MyST markdown documentation}.


\section{Code blocks and outputs}
\label{\detokenize{notebooks:code-blocks-and-outputs}}
\sphinxAtStartPar
Jupyter Book will also embed your code blocks and output in your book.
For example, here’s some sample Matplotlib code:

\begin{sphinxuseclass}{cell}\begin{sphinxVerbatimInput}

\begin{sphinxuseclass}{cell_input}
\begin{sphinxVerbatim}[commandchars=\\\{\}]
\PYG{k+kn}{from} \PYG{n+nn}{matplotlib} \PYG{k+kn}{import} \PYG{n}{rcParams}\PYG{p}{,} \PYG{n}{cycler}
\PYG{k+kn}{import} \PYG{n+nn}{matplotlib}\PYG{n+nn}{.}\PYG{n+nn}{pyplot} \PYG{k}{as} \PYG{n+nn}{plt}
\PYG{k+kn}{import} \PYG{n+nn}{numpy} \PYG{k}{as} \PYG{n+nn}{np}
\PYG{n}{plt}\PYG{o}{.}\PYG{n}{ion}\PYG{p}{(}\PYG{p}{)}
\end{sphinxVerbatim}

\end{sphinxuseclass}\end{sphinxVerbatimInput}
\begin{sphinxVerbatimOutput}

\begin{sphinxuseclass}{cell_output}
\begin{sphinxVerbatim}[commandchars=\\\{\}]
\PYGZlt{}matplotlib.pyplot.\PYGZus{}IonContext at 0x7fbd78524cd0\PYGZgt{}
\end{sphinxVerbatim}

\end{sphinxuseclass}\end{sphinxVerbatimOutput}

\end{sphinxuseclass}
\begin{sphinxuseclass}{cell}\begin{sphinxVerbatimInput}

\begin{sphinxuseclass}{cell_input}
\begin{sphinxVerbatim}[commandchars=\\\{\}]
\PYG{c+c1}{\PYGZsh{} Fixing random state for reproducibility}
\PYG{n}{np}\PYG{o}{.}\PYG{n}{random}\PYG{o}{.}\PYG{n}{seed}\PYG{p}{(}\PYG{l+m+mi}{19680801}\PYG{p}{)}

\PYG{n}{N} \PYG{o}{=} \PYG{l+m+mi}{10}
\PYG{n}{data} \PYG{o}{=} \PYG{p}{[}\PYG{n}{np}\PYG{o}{.}\PYG{n}{logspace}\PYG{p}{(}\PYG{l+m+mi}{0}\PYG{p}{,} \PYG{l+m+mi}{1}\PYG{p}{,} \PYG{l+m+mi}{100}\PYG{p}{)} \PYG{o}{+} \PYG{n}{np}\PYG{o}{.}\PYG{n}{random}\PYG{o}{.}\PYG{n}{randn}\PYG{p}{(}\PYG{l+m+mi}{100}\PYG{p}{)} \PYG{o}{+} \PYG{n}{ii} \PYG{k}{for} \PYG{n}{ii} \PYG{o+ow}{in} \PYG{n+nb}{range}\PYG{p}{(}\PYG{n}{N}\PYG{p}{)}\PYG{p}{]}
\PYG{n}{data} \PYG{o}{=} \PYG{n}{np}\PYG{o}{.}\PYG{n}{array}\PYG{p}{(}\PYG{n}{data}\PYG{p}{)}\PYG{o}{.}\PYG{n}{T}
\PYG{n}{cmap} \PYG{o}{=} \PYG{n}{plt}\PYG{o}{.}\PYG{n}{cm}\PYG{o}{.}\PYG{n}{coolwarm}
\PYG{n}{rcParams}\PYG{p}{[}\PYG{l+s+s1}{\PYGZsq{}}\PYG{l+s+s1}{axes.prop\PYGZus{}cycle}\PYG{l+s+s1}{\PYGZsq{}}\PYG{p}{]} \PYG{o}{=} \PYG{n}{cycler}\PYG{p}{(}\PYG{n}{color}\PYG{o}{=}\PYG{n}{cmap}\PYG{p}{(}\PYG{n}{np}\PYG{o}{.}\PYG{n}{linspace}\PYG{p}{(}\PYG{l+m+mi}{0}\PYG{p}{,} \PYG{l+m+mi}{1}\PYG{p}{,} \PYG{n}{N}\PYG{p}{)}\PYG{p}{)}\PYG{p}{)}


\PYG{k+kn}{from} \PYG{n+nn}{matplotlib}\PYG{n+nn}{.}\PYG{n+nn}{lines} \PYG{k+kn}{import} \PYG{n}{Line2D}
\PYG{n}{custom\PYGZus{}lines} \PYG{o}{=} \PYG{p}{[}\PYG{n}{Line2D}\PYG{p}{(}\PYG{p}{[}\PYG{l+m+mi}{0}\PYG{p}{]}\PYG{p}{,} \PYG{p}{[}\PYG{l+m+mi}{0}\PYG{p}{]}\PYG{p}{,} \PYG{n}{color}\PYG{o}{=}\PYG{n}{cmap}\PYG{p}{(}\PYG{l+m+mf}{0.}\PYG{p}{)}\PYG{p}{,} \PYG{n}{lw}\PYG{o}{=}\PYG{l+m+mi}{4}\PYG{p}{)}\PYG{p}{,}
                \PYG{n}{Line2D}\PYG{p}{(}\PYG{p}{[}\PYG{l+m+mi}{0}\PYG{p}{]}\PYG{p}{,} \PYG{p}{[}\PYG{l+m+mi}{0}\PYG{p}{]}\PYG{p}{,} \PYG{n}{color}\PYG{o}{=}\PYG{n}{cmap}\PYG{p}{(}\PYG{l+m+mf}{.5}\PYG{p}{)}\PYG{p}{,} \PYG{n}{lw}\PYG{o}{=}\PYG{l+m+mi}{4}\PYG{p}{)}\PYG{p}{,}
                \PYG{n}{Line2D}\PYG{p}{(}\PYG{p}{[}\PYG{l+m+mi}{0}\PYG{p}{]}\PYG{p}{,} \PYG{p}{[}\PYG{l+m+mi}{0}\PYG{p}{]}\PYG{p}{,} \PYG{n}{color}\PYG{o}{=}\PYG{n}{cmap}\PYG{p}{(}\PYG{l+m+mf}{1.}\PYG{p}{)}\PYG{p}{,} \PYG{n}{lw}\PYG{o}{=}\PYG{l+m+mi}{4}\PYG{p}{)}\PYG{p}{]}

\PYG{n}{fig}\PYG{p}{,} \PYG{n}{ax} \PYG{o}{=} \PYG{n}{plt}\PYG{o}{.}\PYG{n}{subplots}\PYG{p}{(}\PYG{n}{figsize}\PYG{o}{=}\PYG{p}{(}\PYG{l+m+mi}{10}\PYG{p}{,} \PYG{l+m+mi}{5}\PYG{p}{)}\PYG{p}{)}
\PYG{n}{lines} \PYG{o}{=} \PYG{n}{ax}\PYG{o}{.}\PYG{n}{plot}\PYG{p}{(}\PYG{n}{data}\PYG{p}{)}
\PYG{n}{ax}\PYG{o}{.}\PYG{n}{legend}\PYG{p}{(}\PYG{n}{custom\PYGZus{}lines}\PYG{p}{,} \PYG{p}{[}\PYG{l+s+s1}{\PYGZsq{}}\PYG{l+s+s1}{Cold}\PYG{l+s+s1}{\PYGZsq{}}\PYG{p}{,} \PYG{l+s+s1}{\PYGZsq{}}\PYG{l+s+s1}{Medium}\PYG{l+s+s1}{\PYGZsq{}}\PYG{p}{,} \PYG{l+s+s1}{\PYGZsq{}}\PYG{l+s+s1}{Hot}\PYG{l+s+s1}{\PYGZsq{}}\PYG{p}{]}\PYG{p}{)}\PYG{p}{;}
\end{sphinxVerbatim}

\end{sphinxuseclass}\end{sphinxVerbatimInput}
\begin{sphinxVerbatimOutput}

\begin{sphinxuseclass}{cell_output}
\noindent\sphinxincludegraphics{{notebooks_2_0}.png}

\end{sphinxuseclass}\end{sphinxVerbatimOutput}

\end{sphinxuseclass}
\sphinxAtStartPar
There is a lot more that you can do with outputs (such as including interactive outputs)
with your book. For more information about this, see \sphinxhref{https://jupyterbook.org}{the Jupyter Book documentation}

\begin{sphinxuseclass}{cell}\begin{sphinxVerbatimInput}

\begin{sphinxuseclass}{cell_input}
\begin{sphinxVerbatim}[commandchars=\\\{\}]
\PYG{k+kn}{from} \PYG{n+nn}{IPython}\PYG{n+nn}{.}\PYG{n+nn}{display} \PYG{k+kn}{import} \PYG{n}{HTML}
\PYG{n}{HTML}\PYG{p}{(}\PYG{l+s+s1}{\PYGZsq{}}\PYG{l+s+s1}{\PYGZlt{}iframe width=}\PYG{l+s+s1}{\PYGZdq{}}\PYG{l+s+s1}{560}\PYG{l+s+s1}{\PYGZdq{}}\PYG{l+s+s1}{ height=}\PYG{l+s+s1}{\PYGZdq{}}\PYG{l+s+s1}{315}\PYG{l+s+s1}{\PYGZdq{}}\PYG{l+s+s1}{ src=}\PYG{l+s+s1}{\PYGZdq{}}\PYG{l+s+s1}{https://www.youtube.com/embed/S\PYGZus{}f2qV2\PYGZus{}U00?rel=0\PYGZam{}amp;controls=0\PYGZam{}amp;showinfo=0}\PYG{l+s+s1}{\PYGZdq{}}\PYG{l+s+s1}{ frameborder=}\PYG{l+s+s1}{\PYGZdq{}}\PYG{l+s+s1}{0}\PYG{l+s+s1}{\PYGZdq{}}\PYG{l+s+s1}{ allowfullscreen\PYGZgt{}\PYGZlt{}/iframe\PYGZgt{}}\PYG{l+s+s1}{\PYGZsq{}}\PYG{p}{)}
\end{sphinxVerbatim}

\end{sphinxuseclass}\end{sphinxVerbatimInput}
\begin{sphinxVerbatimOutput}

\begin{sphinxuseclass}{cell_output}
\begin{sphinxVerbatim}[commandchars=\\\{\}]
/Users/KSIsleif/opt/anaconda3/lib/python3.9/site\PYGZhy{}packages/IPython/core/display.py:724: UserWarning: Consider using IPython.display.IFrame instead
  warnings.warn(\PYGZdq{}Consider using IPython.display.IFrame instead\PYGZdq{})
\end{sphinxVerbatim}

\begin{sphinxVerbatim}[commandchars=\\\{\}]
\PYGZlt{}IPython.core.display.HTML object\PYGZgt{}
\end{sphinxVerbatim}

\end{sphinxuseclass}\end{sphinxVerbatimOutput}

\end{sphinxuseclass}
\begin{figure}[htbp]
\centering
\capstart

\noindent\sphinxincludegraphics[width=200\sphinxpxdimen]{{logo}.png}
\caption{This is a caption in \sphinxstylestrong{Markdown}!}\label{\detokenize{notebooks:markdown-fig}}\end{figure}



\sphinxAtStartPar
\sphinxhref{https://youtu.be/StTqXEQ2l-Y?t=35s}{\sphinxincludegraphics{{Ot5DWAW}.png}}


\chapter{Notebooks with MyST Markdown}
\label{\detokenize{markdown-notebooks:notebooks-with-myst-markdown}}\label{\detokenize{markdown-notebooks::doc}}
\sphinxAtStartPar
Jupyter Book also lets you write text\sphinxhyphen{}based notebooks using MyST Markdown.
See \sphinxhref{https://jupyterbook.org/file-types/myst-notebooks.html}{the Notebooks with MyST Markdown documentation} for more detailed instructions.
This page shows off a notebook written in MyST Markdown.


\section{An example cell}
\label{\detokenize{markdown-notebooks:an-example-cell}}
\sphinxAtStartPar
With MyST Markdown, you can define code cells with a directive like so:

\begin{sphinxuseclass}{cell}\begin{sphinxVerbatimInput}

\begin{sphinxuseclass}{cell_input}
\begin{sphinxVerbatim}[commandchars=\\\{\}]
\PYG{n+nb}{print}\PYG{p}{(}\PYG{l+m+mi}{2} \PYG{o}{+} \PYG{l+m+mi}{2}\PYG{p}{)}
\end{sphinxVerbatim}

\end{sphinxuseclass}\end{sphinxVerbatimInput}
\begin{sphinxVerbatimOutput}

\begin{sphinxuseclass}{cell_output}
\begin{sphinxVerbatim}[commandchars=\\\{\}]
4
\end{sphinxVerbatim}

\end{sphinxuseclass}\end{sphinxVerbatimOutput}

\end{sphinxuseclass}
\sphinxAtStartPar
When your book is built, the contents of any \sphinxcode{\sphinxupquote{\{code\sphinxhyphen{}cell\}}} blocks will be
executed with your default Jupyter kernel, and their outputs will be displayed
in\sphinxhyphen{}line with the rest of your content.


\sphinxstrong{See also:}
\nopagebreak


\sphinxAtStartPar
Jupyter Book uses \sphinxhref{https://jupytext.readthedocs.io/en/latest/}{Jupytext} to convert text\sphinxhyphen{}based files to notebooks, and can support \sphinxhref{https://jupyterbook.org/file-types/jupytext.html}{many other text\sphinxhyphen{}based notebook files}.




\section{Create a notebook with MyST Markdown}
\label{\detokenize{markdown-notebooks:create-a-notebook-with-myst-markdown}}
\sphinxAtStartPar
MyST Markdown notebooks are defined by two things:
\begin{enumerate}
\sphinxsetlistlabels{\arabic}{enumi}{enumii}{}{.}%
\item {} 
\sphinxAtStartPar
YAML metadata that is needed to understand if / how it should convert text files to notebooks (including information about the kernel needed).
See the YAML at the top of this page for example.

\item {} 
\sphinxAtStartPar
The presence of \sphinxcode{\sphinxupquote{\{code\sphinxhyphen{}cell\}}} directives, which will be executed with your book.

\end{enumerate}

\sphinxAtStartPar
That’s all that is needed to get started!


\section{Quickly add YAML metadata for MyST Notebooks}
\label{\detokenize{markdown-notebooks:quickly-add-yaml-metadata-for-myst-notebooks}}
\sphinxAtStartPar
If you have a markdown file and you’d like to quickly add YAML metadata to it, so that Jupyter Book will treat it as a MyST Markdown Notebook, run the following command:

\begin{sphinxVerbatim}[commandchars=\\\{\}]
\PYG{n}{jupyter}\PYG{o}{\PYGZhy{}}\PYG{n}{book} \PYG{n}{myst} \PYG{n}{init} \PYG{n}{path}\PYG{o}{/}\PYG{n}{to}\PYG{o}{/}\PYG{n}{markdownfile}\PYG{o}{.}\PYG{n}{md}
\end{sphinxVerbatim}

\begin{sphinxthebibliography}{HdHPK14}
\bibitem[HdHPK14]{markdown:id3}
\sphinxAtStartPar
Christopher Ramsay Holdgraf, Wendy de Heer, Brian N. Pasley, and Robert T. Knight. Evidence for Predictive Coding in Human Auditory Cortex. In \sphinxstyleemphasis{International Conference on Cognitive Neuroscience}. Brisbane, Australia, Australia, 2014. Frontiers in Neuroscience.
\end{sphinxthebibliography}







\renewcommand{\indexname}{Index}
\printindex
\end{document}